\documentclass[a4paper,11pt]{article}
\usepackage[utf8]{inputenc}
\usepackage[T1]{fontenc}
\usepackage[right=2cm,left=2cm,bottom=2cm,top=2.5cm,headheight=2em]{geometry}
\usepackage[ddmmyyyy]{datetime}
\usepackage{xcolor}
\usepackage[colorlinks,linkcolor=black,urlcolor=blue]{hyperref}
\usepackage{listings}
\usepackage{lstautogobble}
\usepackage{tabularx}
\usepackage{csvsimple}
\usepackage{graphicx}
\usepackage{fancyhdr}
\usepackage{pgf}
\usepackage{pgffor}
\usepackage{mdframed}

% sans-serif font ("Computer Modern Sans") instead of roman font
\renewcommand{\rmdefault}{\sfdefault}
% "CourrierNew" font (for code listing)
\renewcommand{\ttdefault}{pcr}

\lstset{
	basicstyle=\small\ttfamily,
	backgroundcolor=\color{white},
	breaklines=true,
	breakatwhitespace=true,
	tabsize=2,
	frame=single,
	rulecolor=\color{black},
	keywordstyle=\color{blue}\bfseries,
	stringstyle=\color{orange},
	showstringspaces=false,
	commentstyle=\footnotesize\color{darkgreen},
	keepspaces=true,
	extendedchars=true,
	numbers=none,
	numberstyle=\tiny\color{lightgray},
	stepnumber=1,
	escapeinside={(@}{@)},
	autogobble=true,
	literate=
		{á}{{\'a}}1 {é}{{\'e}}1 {í}{{}}1 {ó}{{\'o}}1 {ú}{{\'u}}1
		{Á}{{\'A}}1 {É}{{\'E}}1 {Í}{{\'I}}1 {Ó}{{\'O}}1 {Ú}{{\'U}}1
		{à}{{\`a}}1 {è}{{\`e}}1 {ì}{{\`i}}1 {ò}{{\`o}}1 {ù}{{\`u}}1
		{À}{{\`A}}1 {È}{{\'E}}1 {Ì}{{\`I}}1 {Ò}{{\`O}}1 {Ù}{{\`U}}1
		{ä}{{\"a}}1 {ë}{{\"e}}1 {ï}{{\"i}}1 {ö}{{\"o}}1 {ü}{{\"u}}1
		{Ä}{{\"A}}1 {Ë}{{\"E}}1 {Ï}{{\"I}}1 {Ö}{{\"O}}1 {Ü}{{\"U}}1
		{â}{{\^a}}1 {ê}{{\^e}}1 {î}{{\^i}}1 {ô}{{\^o}}1 {û}{{\^u}}1
		{Â}{{\^A}}1 {Ê}{{\^E}}1 {Î}{{\^I}}1 {Ô}{{\^O}}1 {Û}{{\^U}}1
		{œ}{{\oe}}1 {Œ}{{\OE}}1 {æ}{{\ae}}1 {Æ}{{\AE}}1 {ß}{{\ss}}1
		{ç}{{\c c}}1 {Ç}{{\c C}}1 {ø}{{\o}}1 {å}{{\r a}}1 {Å}{{\r A}}1
		{€}{{e}}1 {£}{{\pounds}}1 {«}{{\guillemotleft}}1
		{»}{{\guillemotright}}1 {ñ}{{\~n}}1 {Ñ}{{\~N}}1 {¿}{{?`}}1
}


\pagestyle{fancy}
\renewcommand{\headrulewidth}{0pt}
\renewcommand{\footrulewidth}{0pt}
\fancyhf[CLRHF]{}
\fancyhf[LH]{\includegraphics[height=\dimexpr(\headheight-.5em)\relax]
                             {logo_agenium_scale}}
\fancyhf[CF]{\thepage}

\date{\today}
\author{\includegraphics[width=20em]{logo_agenium_scale}}
\title{GROMACS/NSIMD Benchmarks __ENV_TITLE__}

\newcommand{\gromacs}{GROMACS}
\newcommand{\nsimd}{NSIMD}
\newcommand{\cpu}{CPU}
\newcommand{\ageniumscale}{AGENIUM SCALE}

\begin{document}
\maketitle%
\vspace*{\fill}%
\noindent%
\textbf{\ageniumscale{}}\\
Rue Noetzlin\\
Batiment 660, Digiteo Labs\\
91 190 Gif-sur-Yvette\\
01 69 15 32 32\\
\href{mailto://contact@numscale.com}{contact@numscale.com}\\
\href{https://agenium-scale.com}{https://agenium-scale.com}\\
\thispagestyle{empty}
\clearpage
%
\thispagestyle{empty}
\tableofcontents
\newpage

\subsection*{About}%
\label{sec:about}

This document is a \textbf{technical} report of the \gromacs{} benchmarks
performed by \ageniumscale{} on an Intel Skylake AVX-512 capable \cpu. The aim
is to give a quick idea of how vector code using \nsimd{} performs against
other versions.

The document was automatically generated on \today. No human was involved in
running the benchmarks. There are many benchmarks and not all of them can be
checked by a human. Therefore if you find weird results, fell free to contact
\ageniumscale{} at \href{mailto://contact@numscale.com}{contact@numscale.com}.
If you want us to run benchmarks on a machine you own, feel free to contact us
at \href{mailto://contact@numscale.com}{contact@numscale.com} to see whether
and how we can help you.

This document is meant to be read by software developers. The explanations
provided in the following sections are not intended to be detailed. We assume
that the reader has the knowledge required to understand the present document.
If you have any relevant question feel free to contact us at
\href{mailto://contact@numscale.com}{contact@numscale.com}.

\begin{mdframed}{
  \textbf{Disclaimer}

  All the information, including technical and engineering data, processes,
  and results, presented in this document has been prepared carefully in
  order to present an accurate vision of NSIMD's performance. However, the
  reader is informed that the hardware and software environment may affect
  NSIMD's performance and present results that are distinct from those
  presented herein.

  Thus, \ageniumscale{} does not guarantee in any way the accuracy or
  completeness of the results presented, which are provided for illustrative
  purposes only. The terms used in this document shall not be construed as
  offering any guarantee of result, purpose, and more generally no warranty of
  any kind.

  For more details on our commitments, we refer you to the NSIMD license
  agreement which sets out the scope of our commitments.
}\end{mdframed}

\section{Setup}%
\label{sec:setup}

\subsection{\gromacs{} Tooling}

The benchmarks were done using a fork of \gromacs{} version 2019.3 published on
June 2019 modified to make use of the
\href{https://github.com/agenium-scale/nsimd#nsimd}{\nsimd{}} library. You can
find the source in the
\href{https://github.com/agenium-scale/gromacs/tree/nsimd-translate}
{\texttt{nsimd-translate}} branch of the Git repository
(\href{https://github.com/agenium-scale/gromacs}
{https://github.com/agenium-scale/gromacs}).

\subsection{Benchmark Organisation}

Each comparison requires specific \gromacs{} binaries that have been tailored
for the vector instruction set chosen. We compiled a set of binaries using the
already provided code and a set using \nsimd{}.

The next chapter presents graphs comparing each performance counter coming from
the different runs of the \gromacs{} binaries each of which compiled with its
own set of SIMD instructions.  In the last chapter, the log reports made by
\gromacs{} are given below.

\subsection{Protocol}

For each SIMD extension, \gromacs{}'s \texttt{mdrun} is run as follows on
the ``water" simulation containing 1,536,000 particles. It can be found
at \href{https://ftp.gromacs.org/pub/benchmarks/water_GMX50_bare.tar.gz}.

\begin{lstlisting}
gmx grompp -f pme.mdp
gmx convert-tpr -nsteps 1000
gmx mdrun -nt __NTHREADS__
\end{lstlisting}

Note that the goal of this document is \emph{not} to measures the performances
of a particular compiler or CPU. The \emph{only} goal is to measure NSIMD
performances against \gromacs{}'s native SIMD code. Therefore we do not need
to run \gromacs{} over all cores or over several nodes. We limit the number
of threads to __NTHREADS__ so that each has enough work and no time is spent
on waiting each other.

\subsubsection{CPU}
\lstinputlisting{cpu.info}

\subsubsection{RAM}
\lstinputlisting{mem.info}

\subsubsection{System}
\lstinputlisting{uname.info}

\subsubsection{Compiler}
\lstinputlisting{compiler.info}

\subsubsection{C standard library}
\lstinputlisting{libc.info}
 
\section{Comparison graphs}
\input{comparison-graphs.tex}

\newpage
\section{GROMACS raw logs}

\pgfkeyssetvalue{none}{no SIMD}
\pgfkeyssetvalue{sse2}{SSE 2}
\pgfkeyssetvalue{sse42}{SSE 4.2}
\pgfkeyssetvalue{avx}{AVX}
\pgfkeyssetvalue{avx2}{AVX2}
\pgfkeyssetvalue{avx512-skylake}{AVX-512 (Skylake version)}
\pgfkeyssetvalue{avx512-knl}{AVX-512 (KNL version)}
\pgfkeyssetvalue{neon128}{NEON128}
\pgfkeyssetvalue{aarch64}{AArch64}

\foreach \ext in {none,sse2,sse42,avx,avx2,avx512-skylake,neon128,aarch64}{
  \IfFileExists{\ext-gromacs.log}{%
    \pgfkeysgetvalue{\ext}{\prettyext}%
    \subsection{GROMACS raw log with \prettyext{}}
    \lstinputlisting{\ext-gromacs.log}
    \newpage
  }{}
  \IfFileExists{nsimd-\ext-gromacs.log}{%
    \pgfkeysgetvalue{\ext}{\prettyext}%
    \subsection{GROMACS raw log with \prettyext{} through NSIMD}
    \lstinputlisting{nsimd-\ext-gromacs.log}
    \newpage
  }{}
}

\end{document}
