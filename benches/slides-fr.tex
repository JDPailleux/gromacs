\documentclass[shrink, compress, mathserif, 10pt, xcolor=dvipsnames,
               aspectratio=169]{beamer}
\usetheme{Scale}

\usepackage{textcomp}
%%\usepackage[french]{babel}
\usepackage[T1]{fontenc}
\usepackage[utf8]{inputenc}
\usepackage{helvet}

\title{Bibliothèque NSIMD}
\subtitle{Application à GROMACS, le logiciel de simulation en dynamique
          moléculaire}
\date{\today}
\author{Agenium Scale}

\begin{document}

\begin{frame}[plain]
  \maketitle
\end{frame}

\begin{frame}{NSIMD}
  \begin{itemize}
    \item
      NSIMD Bibliothèque de calcul C, C++98 jusqu'à C++20 open source offrant
      une abstraction sur les jeux d'instructions SIMD des processeurs.
    \item
      NSIMD supporte les dernières technologies: Intel AVX-512, Arm SVE.
    \item
      NSIMD supporte également les GPUs, NVIDIA et AMD.
    \item
      Réduction drastique des coûts de développement pour l'optimisation
      des codes nécessitant du calcul et de la portabilité.
  \end{itemize}
\end{frame}

\begin{frame}{Application à GROMACS}
  \begin{itemize}
    \item
      Logiciel de simulation en dynamique moléculaire développé initialement
      par l'université de Groningue.
    \item
      Très bien optimisé pour les CPUs en utilisant les extensions SIMD.
    \item
      Un dossier de code pour chaque extension SIMD supportée par GROMACS.
    \item
      NSIMD a été ajoutée comme une ``nouvelle extension SIMD" à GROMACS.
    \item
      Le but est d'avoir les \emph{mêmes} temps d'exécution pour montrer que
      NSIMD n'a pas d'overhead.
  \end{itemize}
\end{frame}

\begin{frame}{Temps d'exécution}
  \begin{center}
    \includegraphics[width=0.75\textwidth]{ns-day-gromacs.pdf}
  \end{center}

  ``ns/day" signifie nombre de nanosecondes simulées par 24h de calcul.
\end{frame}

\begin{frame}{Nombre de lignes de code}
  \begin{center}
    \includegraphics[width=0.75\textwidth]{simd-loc-gromacs.pdf}
  \end{center}
\end{frame}

\begin{frame}{Conclusion et liens}
  \begin{itemize}
    \item
      Diminuer du nombre de lignes de code pour la partie SIMD de GROMACS.
    \item
      NSIMD n'a pas d'overhead pour les codes de calcul scientiques et
      industriels.
    \item
      Augmentation de la lisibilité et de la maintenabilité du code.  
  \end{itemize}

  \begin{enumerate}
    \item GROMACS: \url{http://www.gromacs.org}
    \item NSIMD: \url{https://github.com/agenium-scale/nsimd}
    \item GROMACS with NSIMD: \url{https://github.com/agenium-scale/gromacs}
    \item AGENIUM SCALE: \url{https://www.agenium-scale.com}
  \end{enumerate}
\end{frame}

\end{document}
