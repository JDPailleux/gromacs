\documentclass[a4paper,11pt]{article}
\usepackage[utf8]{inputenc}
\usepackage[T1]{fontenc}

\usepackage[right=2cm,left=2cm,bottom=2cm,top=2.5cm,headheight=2em]{geometry}
\usepackage[ddmmyyyy]{datetime}
\usepackage{xcolor}
\usepackage[colorlinks,linkcolor=black,urlcolor=blue]{hyperref}
\usepackage{listings}
\usepackage{lstautogobble}
\usepackage{tabularx}
\usepackage{csvsimple}
\usepackage{graphicx}
\usepackage{fancyhdr}
\usepackage{pgf}
\usepackage{pgffor}


% Utilisation de la police sans-serif ("Computer Modern Sans") pour la police roman
\renewcommand{\rmdefault}{\sfdefault}
% Utilisation d'une police "CourrierNew" pour la police monospaced (pour faire un listing manuel)
\renewcommand{\ttdefault}{pcr}

\lstset{
	basicstyle=\small\ttfamily,
	backgroundcolor=\color{white},
	breaklines=true,
	breakatwhitespace=true,
	tabsize=2,
	frame=single,
	rulecolor=\color{black},
	keywordstyle=\color{blue}\bfseries,
	stringstyle=\color{orange},
	showstringspaces=false,
	commentstyle=\footnotesize\color{darkgreen},
	keepspaces=true,
	extendedchars=true,
	numbers=none,
	numberstyle=\tiny\color{lightgray},
	stepnumber=1,
	escapeinside={(@}{@)},
	autogobble=true,
	literate=
		{á}{{\'a}}1 {é}{{\'e}}1 {í}{{}}1 {ó}{{\'o}}1 {ú}{{\'u}}1
		{Á}{{\'A}}1 {É}{{\'E}}1 {Í}{{\'I}}1 {Ó}{{\'O}}1 {Ú}{{\'U}}1
		{à}{{\`a}}1 {è}{{\`e}}1 {ì}{{\`i}}1 {ò}{{\`o}}1 {ù}{{\`u}}1
		{À}{{\`A}}1 {È}{{\'E}}1 {Ì}{{\`I}}1 {Ò}{{\`O}}1 {Ù}{{\`U}}1
		{ä}{{\"a}}1 {ë}{{\"e}}1 {ï}{{\"i}}1 {ö}{{\"o}}1 {ü}{{\"u}}1
		{Ä}{{\"A}}1 {Ë}{{\"E}}1 {Ï}{{\"I}}1 {Ö}{{\"O}}1 {Ü}{{\"U}}1
		{â}{{\^a}}1 {ê}{{\^e}}1 {î}{{\^i}}1 {ô}{{\^o}}1 {û}{{\^u}}1
		{Â}{{\^A}}1 {Ê}{{\^E}}1 {Î}{{\^I}}1 {Ô}{{\^O}}1 {Û}{{\^U}}1
		{œ}{{\oe}}1 {Œ}{{\OE}}1 {æ}{{\ae}}1 {Æ}{{\AE}}1 {ß}{{\ss}}1
		{ç}{{\c c}}1 {Ç}{{\c C}}1 {ø}{{\o}}1 {å}{{\r a}}1 {Å}{{\r A}}1
		{€}{{e}}1 {£}{{\pounds}}1 {«}{{\guillemotleft}}1
		{»}{{\guillemotright}}1 {ñ}{{\~n}}1 {Ñ}{{\~N}}1 {¿}{{?`}}1
}


\pagestyle{fancy}
\renewcommand{\headrulewidth}{0pt}
\renewcommand{\footrulewidth}{0pt}
\fancyhf[CLRHF]{}
\fancyhf[LH]{\includegraphics[height=\dimexpr(\headheight-.5em)\relax]{logo_agenium_scale}}
\fancyhf[CF]{\thepage}

\date{\today}
\author{\includegraphics[width=20em]{logo_agenium_scale}}
\title{GROMACS Benchmarks of NSIMD on Intel Skylake/AVX-512 capable chip}

\newcommand{\gromacs}{GROMACS}
\newcommand{\nsimd}{NSIMD}
\newcommand{\cpu}{CPU}

\begin{document}
\maketitle%
\vspace*{\fill}%
\noindent%
\textbf{AGENIUM SCALE}\\
Rue Noetzlin\\
Batiment 660, Digiteo Labs\\
91 190 Gif-sur-Yvette\\
01 69 15 32 32\\
\href{mailto://contact@numscale.com}{contact@numscale.com}\\
\href{https://www.numscale.com}{https://www.numscale.com}\\
\thispagestyle{empty}
\clearpage
%
\thispagestyle{empty}
\tableofcontents
\newpage

\subsection*{About}%
\label{sec:about}

This document is a \textbf{technical} report of the \gromacs{} benchmarks performed by AGNENIUM SCALE on an Intel Skylake AVX-512 capable \cpu. The aim is to give a quick idea of how vector code using \nsimd{} performs against other versions.

\section{Setup}%
\label{sec:setup}

\subsection{\gromacs{} Tooling}

The benchmarks were done using a fork of \gromacs{} version 2019.3 published on June 2019 modified to make use of the \href{https://github.com/agenium-scale/nsimd#nsimd}{\nsimd{}} library. You can find the source in the \href{https://github.com/agenium-scale/gromacs/tree/nsimd-translate}{\texttt{nsimd-translate}} branch of the Git repository (\href{https://github.com/agenium-scale/gromacs}{https://github.com/agenium-scale/gromacs}).

All benchmarks were done using tools provided by \gromacs{}. For each benchmark, \texttt{gmx tune\_pme} with only one MPI rank was used. For more information about \texttt{gmx tune\_pme} and how to use it, refer to:
\href{http://manual.gromacs.org/documentation/2018/onlinehelp/gmx-tune_pme.html}{http://manual.\allowbreak{}gromacs.org/documentation/2018/onlinehelp/gmx-tune\_pme.html}.

\subsection{Benchmark Organisation}

The report made by \gromacs{} is given below for each tested SIMD extension with and without \nsimd{}. The core information is extracted to ease the comparison between the handcrafted SIMD versions and to one using \nsimd{}.

\subsection{Protocol}

Each comparison requires specific \gromacs{} binaries that have been tailored for the vector instruction set chosen. We compile a set of binaries using the already provided code and a set using \nsimd{}. The performance reports are then gathered for comparison.


\pgfkeyssetvalue{/vector/sse2}{SSE2}
\pgfkeyssetvalue{/vector/sse42}{SSE4.2}
\pgfkeyssetvalue{/vector/avx}{AVX}
\pgfkeyssetvalue{/vector/avx2}{AVX2}
\pgfkeyssetvalue{/vector/avx512-skylake}{AVX512-Skylake}
\pgfkeyssetvalue{/vector/avx512-knl}{AVX512-KNL}
\pgfkeyssetvalue{/vector/nsimd-sse2}{NSIMD -- SSE2}
\pgfkeyssetvalue{/vector/nsimd-sse42}{NSIMD -- SSE4.2}
\pgfkeyssetvalue{/vector/nsimd-avx}{NSIMD -- AVX}
\pgfkeyssetvalue{/vector/nsimd-avx2}{NSIMD -- AVX2}
\pgfkeyssetvalue{/vector/nsimd-avx512-skylake}{NSIMD -- AVX512-Skylake}
\pgfkeyssetvalue{/vector/nsimd-avx512-knl}{NSIMD -- AVX512-KNL}

\renewcommand{\subsectionmark}[1]{\markright{#1}}
\fancyhf[HR]{\rightmark}

\section{Benchmarks}

\foreach \ext in {sse2,sse42,avx,avx2,avx512}{
  \IfFileExists{../results/\ext-compiler.info}{%
    \pgfkeysgetvalue{/vector/\ext}{\prettyext}%
    \subsection{\prettyext{}}
    \subsubsection{CPU}
    \lstinputlisting{../results/\ext-cpu.info}

    \subsubsection{RAM}
    \lstinputlisting{../results/\ext-mem.info}

    \subsubsection{System}
    \lstinputlisting{../results/\ext-system.info}

    \subsubsection{Compiler}
    \lstinputlisting{../results/\ext-compiler.info}

    \subsubsection{Linker}
    \lstinputlisting{../results/\ext-ldd.info}

    \subsubsection{Intrinsics Performance Report}
    \lstinputlisting{../results/perf-\ext.out}

    \subsubsection{NSIMD for \prettyext{} Performance Report}
    \lstinputlisting{../results/perf-nsimd-\ext.out}

    \subsubsection{Comparison}

    \foreach \exti in {\ext,nsimd-\ext} {
      \noindent\pgfkeysvalueof{/vector/\exti}
      \vspace{.5em}

      \noindent\begin{tabularx}{\linewidth}{|X|X|X|X|}\hline
      \csvreader[no head,late after line=\\\hline]
                {../results/\exti-data.csv}
                {}
                {\csvcoli & \csvcolii & \csvcoliii & \csvcoliv}
      \end{tabularx}

      \vspace{1em}
    }

    \begin{center}
      \includegraphics[width=\textwidth]{../results/\ext.pdf}
    \end{center}
    \newpage
  }{}
}

\end{document}
